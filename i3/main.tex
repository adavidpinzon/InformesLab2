\documentclass[11pt,twocolumn]{article}
\usepackage{graphicx} % Required for inserting images
\usepackage[tmargin=20mm,bmargin=35mm,lmargin=10mm,rmargin=10mm]{geometry}

\setlength{\columnsep}{0.8cm}
\usepackage[utf8]{inputenc}
\usepackage[T1]{fontenc}
\usepackage[english, spanish, es-noindentfirst, es-tabla]{babel}
\decimalpoint
\usepackage{amsmath,amssymb,lmodern,amsfonts} 
\usepackage{bm,latexsym,amsmath,amssymb,amsfonts,mathrsfs}
\usepackage{float}
\usepackage{mathtools}
\usepackage{fancyhdr}
\usepackage{xcolor}
\usepackage{braket}
\usepackage{float} 
\usepackage{multirow}
\usepackage{tikz}
\usepackage{verbatim}
\usepackage{array, booktabs}
\bibliographystyle{unsrt}
\usetikzlibrary{arrows} 
\usetikzlibrary{decorations.pathmorphing}
\usetikzlibrary{backgrounds}
\usetikzlibrary{fit}
\usetikzlibrary{shadows}
\usetikzlibrary{positioning}
\usepackage{lastpage}
\usepackage{titling}
\usepackage[colorlinks=true,bookmarksopen,bookmarksnumbered,linktocpage]{hyperref}
\title{\textbf{Determinación de la resistividad de dos conductores: Constantán y Cromo-Níquel}}
\author{%
  Juan David Mena Gamboa - 2221886\\
  Nicolás Santiago Espinosa Carrillo - 2240678 \\
                      \\
  \textbf{Grupo:} E1
} 
\date{\today}

\pagestyle{fancy}
\fancyhf{}
\renewcommand{\headrulewidth}{1.0pt}
\renewcommand{\footrulewidth}{1.0pt}
\renewcommand{\footruleskip}{-10pt}
\setlength{\headheight}{2cm}
\lhead{\includegraphics[height=1.5cm]{UIS.png}}
\chead{Universidad Industrial de Santander\\
Escuela de Física}
\rhead{\includegraphics[height=1.5cm]{NewLogo.jpg}}

\usepackage[colorlinks=true,bookmarksopen,bookmarksnumbered,linktocpage]{hyperref}
\hypersetup{linkcolor=blue}
\hypersetup{citecolor=red}

\renewcommand{\thesection}{\Roman{section}} 
\renewcommand{\thesubsection}{\thesection.\Roman{subsection}}
\hyphenpenalty=10000

\graphicspath{{graficas/}}

\begin{document}

\maketitle
\thispagestyle{fancy}

\section{Introducción}

La caracterización de materiales por su comportamiento eléctrico permite clasificarlos como conductores y no conductores, distinción que surge de su capacidad para conducir corriente cuando se les aplica una diferencia de potencial. La conductividad $\sigma$ es una propiedad intrínseca que cuantifica la facilidad con la que los portadores de carga (generalmente electrones) fluyen a través de una sección transversal del material. La resistividad $\rho$ se define como el inverso de la conductividad, $\rho = 1/\sigma$.

La resistencia eléctrica de un conductor depende de su longitud $L$, su área transversal $A$, y del material del cual está hecho. Esta dependencia se expresa mediante la relación:

\begin{equation}
R = \rho \frac{L}{A} \quad [\Omega]
\end{equation}

donde $\rho$ es la resistividad del material, medida en ohm-metros ($\Omega \cdot$m). La resistividad es una constante propia de cada material e independiente de las dimensiones geométricas del conductor, la diferencia de potencial aplicada y la corriente que circula por él.

La ley de Ohm establece que, para un conductor a temperatura constante, la razón entre la diferencia de potencial $\Delta V$ aplicada entre dos puntos del conductor y la corriente eléctrica $I$ que circula por él es constante. Esta constante se define como la resistencia eléctrica entre esos puntos:

\begin{equation}
R = \frac{\Delta V}{I} \quad [\Omega]
\end{equation}

En este laboratorio se determinan experimentalmente las resistividades del constantán y del cromo-níquel mediante dos métodos: (i) medición directa de resistencia utilizando un óhmetro; y (ii) medición indirecta mediante la ley de Ohm, midiendo voltaje y corriente.

\section{Metodología}

El experimento se desarrolló mediante un método inductivo en dos fases, basado en observación sistemática, conocimiento teórico y registro de datos.

\textbf{Fase uno: Medición directa de resistencia.} Se montó el sistema mostrado en la Figura 1, utilizando un óhmetro para medir directamente la resistencia de alambres de constantán y cromo-níquel de diferentes diámetros. Se varió la longitud del conductor desde 5 cm hasta 50 cm en incrementos de 5 cm, registrando la resistencia para cada longitud. Se realizaron mediciones para constantán de diámetros 0.35 mm y 0.4 mm, y para cromo-níquel de los mismos diámetros. Se registró el diámetro de cada alambre y se calculó el área transversal mediante $A = \pi r^2$, donde $r = D/2$ es el radio del alambre.

\textbf{Fase dos: Medición indirecta de resistencia (Ley de Ohm).} Se montó el sistema mostrado en la Figura 2, conectando los alambres a una fuente de alimentación y utilizando multímetros digitales para medir voltaje y corriente. Se mantuvo una corriente constante de $I = 0.025$ A para las mediciones de cromo-níquel, mientras que para constantán se varió la corriente según la configuración. Se registraron los valores de voltaje $\Delta V$ y corriente $I$ para las mismas longitudes utilizadas en la Fase uno. La resistencia se calculó mediante $R = \Delta V/I$ para cada longitud.

Para determinar la resistividad experimentalmente, se construyeron gráficas de $R$ vs $L/A$ para cada material y diámetro. La pendiente de estas gráficas corresponde directamente a la resistividad $\rho$ del material.

\section{Tratamiento de Datos}

Para el tratamiento de datos y medidas estadísticas relativas a cuantificar el error del experimento, se usaron las siguientes fórmulas:

Promedio:
\[ \bar{x} = \frac{\sum x_i}{n} \]

Desviación estándar:
\[ S = \sqrt{\frac{\sum (x_i-\bar{x})^2}{n-1}} \]

Incertidumbre:
\[ \delta = \frac{S}{\sqrt{n}} \]

\subsection*{I. Fase uno: Medición directa de resistencia}

\subsubsection*{Constantán 0.4 mm}

Diámetro: $D = 0.4$ mm $= 4 \times 10^{-4}$ m. Radio: $r = 2 \times 10^{-4}$ m. Área transversal: $A = \pi r^2 = \pi (2 \times 10^{-4})^2 = 1.257 \times 10^{-7}$ m$^2$.

\begin{table}[H]
\centering
\caption{Resistencia directa vs longitud para Constantán 0.4 mm.}
\resizebox{\columnwidth}{!}{%
\begin{tabular}{cccc}
\toprule
\textbf{L [cm]} & \textbf{L [m]} & \textbf{R [$\Omega$]} & \textbf{L/A [m$^{-1}$]} \\
\midrule
5 & 0.05 & 0.6 & 3.98$\times 10^5$ \\
10 & 0.10 & 0.7 & 7.96$\times 10^5$ \\
15 & 0.15 & 0.9 & 1.19$\times 10^6$ \\
20 & 0.20 & 1.1 & 1.59$\times 10^6$ \\
25 & 0.25 & 1.3 & 1.99$\times 10^6$ \\
30 & 0.30 & 1.4 & 2.39$\times 10^6$ \\
35 & 0.35 & 1.8 & 2.79$\times 10^6$ \\
40 & 0.40 & 1.9 & 3.18$\times 10^6$ \\
45 & 0.45 & 2.1 & 3.58$\times 10^6$ \\
50 & 0.50 & 2.3 & 3.98$\times 10^6$ \\
\bottomrule
\end{tabular}%
}
\label{tab:constantan_04_directa}
\end{table}

\subsubsection*{Constantán 0.35 mm}

Diámetro: $D = 0.35$ mm $= 3.5 \times 10^{-4}$ m. Radio: $r = 1.75 \times 10^{-4}$ m. Área transversal: $A = \pi r^2 = \pi (1.75 \times 10^{-4})^2 = 9.621 \times 10^{-8}$ m$^2$.

\begin{table}[H]
\centering
\caption{Resistencia directa vs longitud para Constantán 0.35 mm.}
\resizebox{\columnwidth}{!}{%
\begin{tabular}{cccc}
\toprule
\textbf{L [cm]} & \textbf{L [m]} & \textbf{R [$\Omega$]} & \textbf{L/A [m$^{-1}$]} \\
\midrule
5 & 0.05 & 0.6 & 5.20$\times 10^5$ \\
10 & 0.10 & 0.8 & 1.04$\times 10^6$ \\
15 & 0.15 & 1.0 & 1.56$\times 10^6$ \\
20 & 0.20 & 1.3 & 2.08$\times 10^6$ \\
25 & 0.25 & 1.6 & 2.60$\times 10^6$ \\
30 & 0.30 & 1.9 & 3.12$\times 10^6$ \\
35 & 0.35 & 2.1 & 3.64$\times 10^6$ \\
40 & 0.40 & 2.3 & 4.16$\times 10^6$ \\
45 & 0.45 & 2.7 & 4.68$\times 10^6$ \\
50 & 0.50 & 2.9 & 5.20$\times 10^6$ \\
\bottomrule
\end{tabular}%
}
\label{tab:constantan_035_directa}
\end{table}

\subsubsection*{Cromo-Níquel 0.4 mm}

Diámetro: $D = 0.4$ mm $= 4 \times 10^{-4}$ m. Radio: $r = 2 \times 10^{-4}$ m. Área transversal: $A = 1.257 \times 10^{-7}$ m$^2$.

\begin{table}[H]
\centering
\caption{Resistencia directa vs longitud para Cromo-Níquel 0.4 mm.}
\resizebox{\columnwidth}{!}{%
\begin{tabular}{cccc}
\toprule
\textbf{L [cm]} & \textbf{L [m]} & \textbf{R [$\Omega$]} & \textbf{L/A [m$^{-1}$]} \\
\midrule
5 & 0.05 & 0.4 & 3.98$\times 10^5$ \\
10 & 0.10 & 1.1 & 7.96$\times 10^5$ \\
15 & 0.15 & 1.2 & 1.19$\times 10^6$ \\
20 & 0.20 & 1.9 & 1.59$\times 10^6$ \\
25 & 0.25 & 2.2 & 1.99$\times 10^6$ \\
30 & 0.30 & 2.9 & 2.39$\times 10^6$ \\
35 & 0.35 & 3.2 & 2.79$\times 10^6$ \\
40 & 0.40 & 3.7 & 3.18$\times 10^6$ \\
45 & 0.45 & 3.8 & 3.58$\times 10^6$ \\
50 & 0.50 & 4.3 & 3.98$\times 10^6$ \\
\bottomrule
\end{tabular}%
}
\label{tab:cromoniquel_04_directa}
\end{table}

\subsubsection*{Cromo-Níquel 0.35 mm}

Diámetro: $D = 0.35$ mm $= 3.5 \times 10^{-4}$ m. Radio: $r = 1.75 \times 10^{-4}$ m. Área transversal: $A = 9.621 \times 10^{-8}$ m$^2$.

\begin{table}[H]
\centering
\caption{Resistencia directa vs longitud para Cromo-Níquel 0.35 mm.}
\resizebox{\columnwidth}{!}{%
\begin{tabular}{cccc}
\toprule
\textbf{L [cm]} & \textbf{L [m]} & \textbf{R [$\Omega$]} & \textbf{L/A [m$^{-1}$]} \\
\midrule
5 & 0.05 & 0.3 & 5.20$\times 10^5$ \\
10 & 0.10 & 0.9 & 1.04$\times 10^6$ \\
15 & 0.15 & 1.5 & 1.56$\times 10^6$ \\
20 & 0.20 & 2.2 & 2.08$\times 10^6$ \\
25 & 0.25 & 2.6 & 2.60$\times 10^6$ \\
30 & 0.30 & 3.1 & 3.12$\times 10^6$ \\
35 & 0.35 & 3.7 & 3.64$\times 10^6$ \\
40 & 0.40 & 4.3 & 4.16$\times 10^6$ \\
45 & 0.45 & 4.9 & 4.68$\times 10^6$ \\
50 & 0.50 & 5.5 & 5.20$\times 10^6$ \\
\bottomrule
\end{tabular}%
}
\label{tab:cromoniquel_035_directa}
\end{table}

\noindent\textbf{Procedimiento paso a paso (Fase uno).}
\begin{enumerate}
  \item Para cada material y diámetro, se midió directamente la resistencia $R$ para diferentes longitudes $L$.
  \item Se calculó el área transversal: $A = \pi r^2 = \pi (D/2)^2$, donde $D$ es el diámetro del alambre.
  \item Se calculó la razón $L/A$ para cada medición.
  \item Se construyó la gráfica $R$ vs $L/A$. La pendiente de esta gráfica corresponde a la resistividad $\rho$.
  \item A partir de la relación $R = \rho (L/A)$, la pendiente $m$ de la recta ajustada es: $\rho = m$.
\end{enumerate}

\subsection*{II. Fase dos: Medición indirecta de resistencia (Ley de Ohm)}

\subsubsection*{Constantán 0.4 mm}

\begin{table}[H]
\centering
\caption{Voltaje y corriente vs longitud para Constantán 0.4 mm.}
\resizebox{\columnwidth}{!}{%
\begin{tabular}{ccccc}
\toprule
\textbf{L [cm]} & \textbf{V [V]} & \textbf{I [A]} & \textbf{R [$\Omega$]} & \textbf{L/A [m$^{-1}$]} \\
\midrule
5 & 0.007 & 0.027 & 0.259 & 3.98$\times 10^5$ \\
10 & 0.020 & 0.027 & 0.741 & 7.96$\times 10^5$ \\
15 & 0.033 & -- & -- & 1.19$\times 10^6$ \\
20 & 0.034 & -- & -- & 1.59$\times 10^6$ \\
25 & 0.040 & -- & -- & 1.99$\times 10^6$ \\
30 & 0.044 & -- & -- & 2.39$\times 10^6$ \\
35 & 0.045 & -- & -- & 2.79$\times 10^6$ \\
40 & 0.043 & -- & -- & 3.18$\times 10^6$ \\
45 & 0.049 & -- & -- & 3.58$\times 10^6$ \\
50 & 0.054 & -- & -- & 3.98$\times 10^6$ \\
\bottomrule
\end{tabular}%
}
\label{tab:constantan_04_ohm}
\end{table}

\subsubsection*{Constantán 0.35 mm}

\begin{table}[H]
\centering
\caption{Voltaje vs longitud para Constantán 0.35 mm.}
\resizebox{\columnwidth}{!}{%
\begin{tabular}{cccc}
\toprule
\textbf{L [cm]} & \textbf{V [V]} & \textbf{R [$\Omega$]} & \textbf{L/A [m$^{-1}$]} \\
\midrule
5 & 0.009 & -- & 5.20$\times 10^5$ \\
10 & 0.015 & -- & 1.04$\times 10^6$ \\
15 & 0.023 & -- & 1.56$\times 10^6$ \\
20 & 0.030 & -- & 2.08$\times 10^6$ \\
25 & 0.041 & -- & 2.60$\times 10^6$ \\
30 & 0.046 & -- & 3.12$\times 10^6$ \\
35 & 0.053 & -- & 3.64$\times 10^6$ \\
40 & 0.058 & -- & 4.16$\times 10^6$ \\
45 & 0.064 & -- & 4.68$\times 10^6$ \\
50 & 0.071 & -- & 5.20$\times 10^6$ \\
\bottomrule
\end{tabular}%
}
\label{tab:constantan_035_ohm}
\end{table}

\subsubsection*{Cromo-Níquel 0.35 mm (I = 0.025 A constante)}

\begin{table}[H]
\centering
\caption{Voltaje vs longitud para Cromo-Níquel 0.35 mm con I = 0.025 A.}
\resizebox{\columnwidth}{!}{%
\begin{tabular}{ccccc}
\toprule
\textbf{L [cm]} & \textbf{V [V]} & \textbf{I [A]} & \textbf{R [$\Omega$]} & \textbf{L/A [m$^{-1}$]} \\
\midrule
5 & 0.015 & 0.025 & 0.600 & 5.20$\times 10^5$ \\
10 & 0.034 & 0.025 & 1.360 & 1.04$\times 10^6$ \\
15 & 0.054 & 0.025 & 2.160 & 1.56$\times 10^6$ \\
20 & 0.069 & 0.025 & 2.760 & 2.08$\times 10^6$ \\
25 & 0.082 & 0.025 & 3.280 & 2.60$\times 10^6$ \\
30 & 0.096 & 0.025 & 3.840 & 3.12$\times 10^6$ \\
35 & 0.115 & 0.025 & 4.600 & 3.64$\times 10^6$ \\
40 & 0.121 & 0.025 & 4.840 & 4.16$\times 10^6$ \\
45 & 0.147 & 0.025 & 5.880 & 4.68$\times 10^6$ \\
50 & 0.154 & 0.025 & 6.160 & 5.20$\times 10^6$ \\
\bottomrule
\end{tabular}%
}
\label{tab:cromoniquel_035_ohm}
\end{table}

\subsubsection*{Cromo-Níquel 0.4 mm}

\begin{table}[H]
\centering
\caption{Voltaje vs longitud para Cromo-Níquel 0.4 mm.}
\resizebox{\columnwidth}{!}{%
\begin{tabular}{cccc}
\toprule
\textbf{L [cm]} & \textbf{V [V]} & \textbf{R [$\Omega$]} & \textbf{L/A [m$^{-1}$]} \\
\midrule
5 & 0.020 & -- & 3.98$\times 10^5$ \\
10 & 0.026 & -- & 7.96$\times 10^5$ \\
15 & 0.042 & -- & 1.19$\times 10^6$ \\
20 & 0.052 & -- & 1.59$\times 10^6$ \\
25 & 0.059 & -- & 1.99$\times 10^6$ \\
30 & 0.100 & -- & 2.39$\times 10^6$ \\
35 & 0.108 & -- & 2.79$\times 10^6$ \\
40 & 0.114 & -- & 3.18$\times 10^6$ \\
45 & 0.125 & -- & 3.58$\times 10^6$ \\
50 & 0.134 & -- & 3.98$\times 10^6$ \\
\bottomrule
\end{tabular}%
}
\label{tab:cromoniquel_04_ohm}
\end{table}

\noindent\textbf{Procedimiento paso a paso (Fase dos).}
\begin{enumerate}
  \item Para cada material y diámetro, se midieron voltaje $\Delta V$ y corriente $I$ para diferentes longitudes $L$.
  \item Se calculó la resistencia mediante la ley de Ohm: $R = \Delta V/I$.
  \item Se calculó la razón $L/A$ para cada medición (mismo procedimiento que en Fase uno).
  \item Se construyó la gráfica $R$ vs $L/A$. La pendiente de esta gráfica corresponde a la resistividad $\rho$.
  \item Ejemplo de cálculo: Para Cromo-Níquel 0.35 mm, $L = 5$ cm, $V = 0.015$ V, $I = 0.025$ A:
  \begin{equation}
  R = \frac{\Delta V}{I} = \frac{0.015}{0.025} = 0.600\ \Omega
  \end{equation}
\end{enumerate}

\subsection*{Desarrollo analítico de cálculos}
\subsubsection*{Determinación de $\rho$ por método directo}
De $R=\rho \, L/A$ se obtiene una regresión lineal $R=m(L/A)+b$ donde idealmente $b\approx 0$ y $m=\rho$.
\[
\rho \equiv m = \frac{\sum (x_i-\bar{x})(y_i-\bar{y})}{\sum (x_i-\bar{x})^2},\quad
 x_i=\frac{L_i}{A},\ y_i=R_i.
\]
Ejemplo (Constantán 0.40 mm): con $A=\pi(2\times 10^{-4})^2=1.257\times 10^{-7}$ m$^2$ y los pares $(L/A,R)$ de la Tabla~\ref{tab:constantan_04_directa}, el ajuste produce
\[
\rho=(49.20\pm \delta)\times 10^{-8}\ \Omega\cdot \mathrm{m},\qquad R^2=0.9914.
\]
\subsubsection*{Desviación, incertidumbre y error relativo}
\[
\bar{\rho}=\frac{\sum \rho_j}{n},\quad
S_\rho=\sqrt{\frac{\sum(\rho_j-\bar{\rho})^2}{n-1}},\quad
\delta_\rho=\frac{S_\rho}{\sqrt{n}},\quad
\varepsilon_\mathrm{rel}=\frac{|\bar{\rho}-\rho_\mathrm{ref}|}{\rho_\mathrm{ref}}\times 100\%.
\]
Análogamente para Cromo–Níquel y para el método indirecto (siguiente apartado).
\subsubsection*{Determinación de $\rho$ por método indirecto}
Primero se calcula $R_i=\Delta V_i/I_i$; luego se ajusta $R$ en función de $L/A$ igual que en el método directo, de donde
\[
\rho_\text{Ohm}\equiv m_{R\ \mathrm{vs}\ L/A}.
\]
Ejemplo (Cromo–Níquel 0.35 mm, $I=0.025$ A constante): usando la Tabla~\ref{tab:cromoniquel_035_ohm},
\[
\rho_\text{Ohm}=(117.97\pm \delta)\times 10^{-8}\ \Omega\cdot \mathrm{m},\quad R^2=0.9933.
\]
Finalmente se comparan $\rho_\text{directo}$ y $\rho_\text{Ohm}$ con el valor de referencia bibliográfico.
\section{Análisis de Resultados}

Los resultados experimentales obtenidos permiten determinar la resistividad de los materiales constantán y cromo-níquel mediante dos métodos independientes, validando la relación fundamental $R = \rho (L/A)$ y la ley de Ohm.

\subsection{Relación entre Resistencia y Geometría del Conductor}

Para todas las configuraciones estudiadas, se observa una relación lineal entre la resistencia $R$ y la razón $L/A$, confirmando la validez de la expresión teórica $R = \rho (L/A)$. La Figura \ref{fig:R_vs_L} muestra la dependencia de la resistencia con la longitud del conductor para todos los materiales y diámetros estudiados, donde se aprecia claramente la relación lineal esperada.

\begin{figure}[H]
\centering
\includegraphics[width=0.5\textwidth]{R_vs_L.png}
\caption{Dependencia de la resistencia con la longitud para constantán y cromo-níquel en diferentes diámetros.}
\label{fig:R_vs_L}
\end{figure}

\subsection{Determinación Experimental de la Resistividad - Fase 1}

La resistividad $\rho$ se determina experimentalmente como la pendiente de las gráficas $R$ vs $L/A$. Las Figuras \ref{fig:constantan_04_fase1}, \ref{fig:constantan_035_fase1}, \ref{fig:cromoniquel_04_fase1} y \ref{fig:cromoniquel_035_fase1} muestran los resultados obtenidos mediante medición directa de resistencia.

\begin{figure}[H]
\centering
\includegraphics[width=0.5\textwidth]{constantan_04 mm_fase1.png}
\caption{Resistencia vs $L/A$ para Constantán 0.4 mm - Fase 1 (Medición directa).}
\label{fig:constantan_04_fase1}
\end{figure}

\begin{figure}[H]
\centering
\includegraphics[width=0.5\textwidth]{constantan_035 mm_fase1.png}
\caption{Resistencia vs $L/A$ para Constantán 0.35 mm - Fase 1 (Medición directa).}
\label{fig:constantan_035_fase1}
\end{figure}

\begin{figure}[H]
\centering
\includegraphics[width=0.5\textwidth]{cromo_niquel_04 mm_fase1.png}
\caption{Resistencia vs $L/A$ para Cromo-Níquel 0.4 mm - Fase 1 (Medición directa).}
\label{fig:cromoniquel_04_fase1}
\end{figure}

\begin{figure}[H]
\centering
\includegraphics[width=0.5\textwidth]{cromo_niquel_035 mm_fase1.png}
\caption{Resistencia vs $L/A$ para Cromo-Níquel 0.35 mm - Fase 1 (Medición directa).}
\label{fig:cromoniquel_035_fase1}
\end{figure}

Los valores de resistividad obtenidos mediante medición directa son:
\begin{itemize}
\item Constantán 0.4 mm: $\rho = 49.20 \times 10^{-8}$ $\Omega \cdot$m (error: 0.4\%, $R^2 = 0.9914$)
\item Constantán 0.35 mm: $\rho = 50.38 \times 10^{-8}$ $\Omega \cdot$m (error: 2.8\%, $R^2 = 0.9963$)
\item Cromo-Níquel 0.4 mm: $\rho = 108.30 \times 10^{-8}$ $\Omega \cdot$m (error: 1.5\%, $R^2 = 0.9870$)
\item Cromo-Níquel 0.35 mm: $\rho = 109.39 \times 10^{-8}$ $\Omega \cdot$m (error: 0.6\%, $R^2 = 0.9986$)
\end{itemize}

Estos valores muestran excelente concordancia con los valores teóricos reportados en la bibliografía ($\rho_{\text{constantán}} = 49 \times 10^{-8}$ $\Omega \cdot$m y $\rho_{\text{cromo-níquel}} = 110 \times 10^{-8}$ $\Omega \cdot$m), con errores relativos menores al 3\% y coeficientes de determinación $R^2$ superiores a 0.98, indicando un ajuste lineal muy satisfactorio.

\subsection{Determinación Experimental de la Resistividad - Fase 2}

La Figura \ref{fig:cromoniquel_035_fase2} muestra los resultados obtenidos mediante medición indirecta de resistencia utilizando la ley de Ohm para Cromo-Níquel 0.35 mm.

\begin{figure}[H]
\centering
\includegraphics[width=0.5\textwidth]{cromo_niquel_035 mm_fase2.png}
\caption{Resistencia vs $L/A$ para Cromo-Níquel 0.35 mm - Fase 2 (Ley de Ohm).}
\label{fig:cromoniquel_035_fase2}
\end{figure}

El valor de resistividad obtenido mediante este método es $\rho = 117.97 \times 10^{-8}$ $\Omega \cdot$m (error: 7.2\%, $R^2 = 0.9933$). Aunque el error es mayor que en la Fase 1, el coeficiente de determinación sigue siendo excelente, confirmando la validez del método.

\subsection{Comparación entre Métodos de Medición}

La Figura \ref{fig:comparacion_resistividades} muestra una comparación de los valores de resistividad obtenidos mediante ambos métodos junto con los valores teóricos.

\begin{figure}[H]
\centering
\includegraphics[width=0.5\textwidth]{comparacion_resistividades.png}
\caption{Comparación de resistividades obtenidas experimentalmente mediante ambos métodos.}
\label{fig:comparacion_resistividades}
\end{figure}

Los valores de resistividad obtenidos mediante medición directa (Fase uno) y medición indirecta mediante la ley de Ohm (Fase dos) muestran buena consistencia. Para Cromo-Níquel 0.35 mm, la diferencia entre ambos métodos es de aproximadamente 8\%, lo cual puede atribuirse a las limitaciones instrumentales y a posibles variaciones en las condiciones experimentales entre ambas fases.

\subsection{Fuentes de Error y Limitaciones}

Los errores observados pueden atribuirse a:
\begin{itemize}
\item \textbf{Precisión en la medición del diámetro}: Errores en la determinación del diámetro del alambre afectan directamente el cálculo del área transversal y, por ende, la determinación de $\rho$.
\item \textbf{Variaciones de temperatura}: Cambios en la temperatura durante las mediciones pueden alterar el valor de la resistividad, ya que $\rho$ depende de la temperatura.
\item \textbf{Homogeneidad del material}: Se asume que el material es uniforme y homogéneo a lo largo de toda su longitud.
\item \textbf{Precisión instrumental}: Resolución limitada de los multímetros y óhmetros empleados.
\item \textbf{Contactos eléctricos}: Resistencia de contacto en los puntos de conexión puede introducir errores sistemáticos.
\end{itemize}

\section{Conclusiones}

El estudio experimental de la resistividad de conductores ha permitido validar los principios fundamentales de la conducción eléctrica y determinar experimentalmente las resistividades del constantán y del cromo-níquel mediante dos métodos independientes.

\subsection{Validación de Principios Teóricos}

\begin{itemize}
\item La relación lineal entre resistencia y la razón $L/A$ se confirma experimentalmente para todos los materiales y diámetros estudiados, validando la expresión $R = \rho (L/A)$.
\item La ley de Ohm se verifica experimentalmente, permitiendo determinar la resistencia de forma indirecta mediante la medición de voltaje y corriente.
\item La resistividad se confirma como una constante propia de cada material, independiente de las dimensiones geométricas del conductor.
\end{itemize}

\subsection{Determinación de la Resistividad}

\begin{itemize}
\item Los valores experimentales de resistividad obtenidos para constantán y cromo-níquel mediante ambos métodos muestran excelente concordancia con los valores teóricos reportados en la bibliografía.
\item Para constantán, los valores experimentales ($49.20 - 50.38 \times 10^{-8}$ $\Omega \cdot$m) se encuentran muy cerca del valor teórico ($49 \times 10^{-8}$ $\Omega \cdot$m), con errores relativos menores al 3\%.
\item Para cromo-níquel, los valores experimentales ($108.30 - 117.97 \times 10^{-8}$ $\Omega \cdot$m) también muestran buena concordancia con el valor teórico ($110 \times 10^{-8}$ $\Omega \cdot$m), con errores relativos menores al 8\%.
\item Las pequeñas diferencias entre valores experimentales y teóricos pueden atribuirse principalmente a limitaciones instrumentales, variaciones de temperatura durante las mediciones y precisión en la medición del diámetro del alambre.
\item La metodología empleada proporciona una base sólida para la caracterización experimental de propiedades eléctricas de materiales, validando ambos métodos de medición.
\end{itemize}

\subsection{Influencia de la Geometría}

\begin{itemize}
\item Se observa que la resistencia aumenta al incrementar la longitud del conductor, como se espera teóricamente.
\item Se confirma que al aumentar el área transversal, la resistencia del conductor disminuye.
\item Estos resultados validan la dependencia de la resistencia con las dimensiones geométricas del conductor.
\end{itemize}

\section{Anexos}
% Imágenes adicionales presentes en la carpeta Anexos
\begin{figure}[H]\centering
\resizebox{0.4\columnwidth}{!}{\includegraphics{Anexos/Imagen de WhatsApp 2025-11-15 a las 17.31.02_c60502fb.jpg}}\hfill
\resizebox{0.4\columnwidth}{!}{\includegraphics{Anexos/Imagen de WhatsApp 2025-11-15 a las 17.31.02_fbc66099.jpg}}
\caption{Evidencias adicionales del montaje (1).}
\end{figure}
\begin{figure}[H]\centering
\resizebox{0.4\columnwidth}{!}{\includegraphics{Anexos/Imagen de WhatsApp 2025-11-15 a las 17.31.03_a2d3b996.jpg}}\hfill
\resizebox{0.4\columnwidth}{!}{\includegraphics{Anexos/Imagen de WhatsApp 2025-11-15 a las 17.31.03_f7dac5e2.jpg}}
\caption{Evidencias adicionales del montaje (2).}
\end{figure}

\begin{thebibliography}{9}

\bibitem{Serway} Serway, R. A., \& Jewett, J. W. (2018). \textit{Physics for Scientists and Engineers} (10th ed.). Cengage Learning.

\bibitem{Young} Young, H. D., Freedman, R. A., \& Ford, A. L. (2009). \textit{University Physics with Modern Physics} (12th ed.). Pearson Addison-Wesley.

\bibitem{Giancoli} Giancoli, D. C. (2009). \textit{Physics for Scientists and Engineers with Modern Physics} (4th ed.). Pearson Prentice Hall.

\bibitem{Resistivity} Callister, W. D., \& Rethwisch, D. G. (2018). \textit{Materials Science and Engineering: An Introduction} (10th ed.). John Wiley \& Sons.

\end{thebibliography}


\end{document}
