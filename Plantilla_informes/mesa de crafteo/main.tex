\documentclass[11pt,twocolumn]{article}
\usepackage{graphicx} % Required for inserting images
\usepackage[tmargin=20mm,bmargin=35mm,lmargin=10mm,rmargin=10mm]{geometry}

\setlength{\columnsep}{0.8cm}
\usepackage[utf8]{inputenc}
\usepackage[T1]{fontenc}
\usepackage[english, spanish, es-noindentfirst, es-tabla]{babel}
\decimalpoint
\usepackage{amsmath,amssymb,lmodern,amsfonts} 
\usepackage{bm,latexsym,amsmath,amssymb,amsfonts,mathrsfs}
\usepackage{float}
\usepackage{mathtools}
\usepackage{fancyhdr}
\usepackage{xcolor}
\usepackage{braket}
\usepackage{float} 
\usepackage{multirow}
\usepackage{tikz}
\usepackage{verbatim}
\usepackage{array, booktabs}
\bibliographystyle{unsrt}
\usetikzlibrary{arrows} 
\usetikzlibrary{decorations.pathmorphing}
\usetikzlibrary{backgrounds}
\usetikzlibrary{fit}
\usetikzlibrary{shadows}
\usetikzlibrary{positioning}
\usepackage{lastpage}
\usepackage{titling}
\usepackage[colorlinks=true,bookmarksopen,bookmarksnumbered,linktocpage]{hyperref}
\title{\textbf{Laboratorio i2}}
\author{%
  Juan David Mena Gamboa\\
  Nicolás Santiago Espinosa Carrillo - 2240678 \\
                      \\
  \textbf{Grupo:} E1
} 
\date{\today}

\pagestyle{fancy}
\fancyhf{}
\renewcommand{\headrulewidth}{1.0pt}
\renewcommand{\footrulewidth}{1.0pt}
\renewcommand{\footruleskip}{-10pt}
\setlength{\headheight}{2cm}
\lhead{\includegraphics[height=1.5cm]{UIS.png}}
\chead{Universidad Industrial de Santander\\
Escuela de Física}
\rhead{\includegraphics[height=1.5cm]{NewLogo.jpg}}

\usepackage[colorlinks=true,bookmarksopen,bookmarksnumbered,linktocpage]{hyperref}
\hypersetup{linkcolor=blue}
\hypersetup{citecolor=red}

\renewcommand{\thesection}{\Roman{section}} 
\renewcommand{\thesubsection}{\thesection.\Roman{subsection}}
\hyphenpenalty=10000

\begin{document}

\maketitle
\thispagestyle{fancy}

\section{Introducción}

La relación entre el voltaje, la corriente y la resistencia en un material constituye uno de los principios fundamentales de la física y la ingeniería eléctrica. Esta relación está descrita por la \textbf{Ley de Ohm}, formulada por Georg Simon Ohm en el siglo XIX, la cual establece que la corriente eléctrica $I$ que circula por un conductor es directamente proporcional al voltaje $V$ aplicado e inversamente proporcional a su resistencia $R$:

\begin{equation}
I = \frac{V}{R}.
\end{equation}

De esta ecuación se desprende también que:
\begin{equation}
V = RI,
\end{equation}
cuando la resistencia se mantiene constante. Sin embargo, no todos los materiales siguen un comportamiento estrictamente lineal. Los \textit{materiales óhmicos} presentan una relación lineal entre $V$ e $I$, mientras que los \textit{materiales no-óhmicos} exhiben dependencias no lineales, como ocurre en dispositivos electrónicos y lámparas incandescentes.  

El estudio experimental de estas relaciones permite responder preguntas clave: ¿qué materiales cumplen la Ley de Ohm y cuáles no?, ¿cómo se cuantifica la variación de la resistencia en función de la corriente y del voltaje?, ¿qué factores externos, como la temperatura o las características internas del conductor, pueden modificar el comportamiento observado?  

Entre las variables que afectan el fenómeno se encuentran la \textbf{longitud y sección transversal del material}, la \textbf{temperatura} alcanzada por el conductor bajo el paso de corriente, y la \textbf{naturaleza del material} (metales, semiconductores o dispositivos específicos). Comprender estas dependencias es esencial para el diseño de circuitos eléctricos y la caracterización de componentes en aplicaciones tecnológicas.  

En este experimento se buscó corroborar la validez de la Ley de Ohm para materiales óhmicos, caracterizar materiales no-óhmicos, y establecer mediante mediciones experimentales las relaciones cuantitativas entre voltaje, corriente y resistencia.

\section{Metodología}

El procedimiento experimental se desarrolló en tres fases, utilizando fuentes de voltaje de corriente continua, resistencias, un reóstato, un bombillo incandescente, un amperímetro y un voltímetro (multímetro digital). 

\subsection*{Fase 1: Relación voltaje--corriente en material óhmico}
\begin{enumerate}
    \item Se realizó el montaje del circuito con una resistencia nominal de 60 \(\Omega\) conectada en serie con un amperímetro y en paralelo con un voltímetro.
    \item Se corroboró el valor de la resistencia con ayuda de un multímetro en la escala de ohmios.
    \item Se aplicó un voltaje variable en el rango de $7,41 \, \text{V}$ a $55,34 \, \text{V}$, aumentando gradualmente.
    \item Se registraron al menos diez valores de corriente correspondientes a cada voltaje aplicado.
    \item Los datos se tabularon para posteriormente graficar $V$ vs. $I$.
\end{enumerate}

\subsection*{Fase 2: Relación corriente--resistencia con voltaje constante}
\begin{enumerate}
    \item Se realizó un montaje similar al de la fase anterior, reemplazando la resistencia por un reóstato ajustable.
    \item Se fijó un voltaje constante de $40 \text{V}$ proporcionado por la fuente de corriente alterna.
    \item Se variaron los valores de resistencia, en el rango de $60 \Omega$ a $80 \Omega$.
    \item Se registró la corriente correspondiente a cada resistencia y se construyó la gráfica $I$ vs. $R$.
\end{enumerate}

\subsection*{Fase 3: Relación voltaje--corriente en material no-óhmico}
\begin{enumerate}
    \item Se sustituyó la resistencia nominal por un bombillo incandescente en el montaje de la fase 1.
    \item Se variaron los valores de voltaje aplicados en un rango de $45 \text{V}$ a $95 \text{V}$.
    \item Se midió la corriente en cada caso con el amperímetro y se construyó la gráfica $V$ vs. $I$.
    \item Se comparó el comportamiento obtenido con el de los materiales óhmicos.
\end{enumerate}

\section{Tratamiento de datos}

\subsection*{Fórmulas estadísticas}
Promedio:
\[
\bar{X}=\frac{1}{n}\sum_{i=1}^{n} X_i
\]

Desviación estándar:
\[
S_X=\sqrt{\frac{1}{n-1}\sum_{i=1}^{n}(X_i-\bar{X})^2}
\]

Incertidumbre:
\[
\delta_{\bar{X}}=\frac{S_X}{\sqrt{n}}
\]

Porcentaje de error relativo:
\[
\%\mathrm{error}_i = \frac{\lvert R_i - R_{\mathrm{ref}}\rvert}{R_{\mathrm{ref}}}\times 100\%
\]
(El promedio de los porcentajes de error se reporta como \(\overline{\%\mathrm{error}}\).)

\subsection*{Fase 1 — Resistencia constante en material óhmico $R_{\mathrm{ref}} = 60\ \Omega$}

Se registraron los pares $(V,I)$ y se calculó la resistencia implícita por medición \(R_i=V_i/I_i\). Con las \(n=10\) mediciones se calcularon la media, la desviación estándar muestral y la incertidumbre del promedio. También se calculó el porcentaje de error relativo frente a \(60\ \Omega\) y su promedio.

\begin{table}[h]
\centering
\caption{Fase 1: Mediciones y cálculos.}
\label{tab:fase1}


\begin{tabular}{r r r r}
\hline
$V\ [\mathrm{V}]$ & $I\ [\mathrm{A}]$ & $R=V/I\ [\Omega]$ & Incertidumbre resistencia \\
\hline
7.41   & 0.120 & 61.75  & $\pm 0.22$ \\
11.86  & 0.190 & 62.42  & $\pm 0.22$ \\
17.84  & 0.290 & 61.52  & $\pm 0.22$ \\
21.57  & 0.350 & 61.63  & $\pm 0.22$ \\
27.02  & 0.440 & 61.41  & $\pm 0.22$ \\
32.15  & 0.530 & 60.66  & $\pm 0.22$ \\
40.60  & 0.672 & 60.42  & $\pm 0.22$ \\
46.31  & 0.761 & 60.85  & $\pm 0.22$ \\
51.02  & 0.824 & 61.92  & $\pm 0.22$ \\
55.34  & 0.918 & 60.28  & $\pm 0.22$ \\
\hline
\end{tabular}
\end{table}

\paragraph{Resultados estadísticos calculados}
\[
\overline{R} = 61.29\ \Omega
\]
\[
S_R = 0.70\ \Omega 
\]
\[
\delta_{\overline{R}} = \frac{S_R}{\sqrt{n}} = 0.22\ \Omega
\]

El promedio de los porcentajes de error es:
\[
\overline{\%\mathrm{error}} = 2.14\ \%.
\]

\subsection*{Fase 2 — Voltaje constante $V=40\ \mathrm{V}$ en un material óhmico.}

Se registraron las corrientes medidas \(I_{\mathrm{med}}\) para distintas resistencias \(R\). Se calculó la corriente teórica \(I_{\mathrm{teo}}=V/R\) y el porcentaje de error relativo para cada medición.


\begin{table}[h]
\centering
\caption{Fase 2: Mediciones y cálculos.}

\label{tab:fase2}

\begin{tabular}{r r r r r}
\hline
$I_{\mathrm{med}}\ [\mathrm{A}]$ & $R\ [\Omega]$ & $I_{\mathrm{teo}}=\dfrac{40}{R}\ [\mathrm{A}]$ & Incertidumbre $[\mathrm{A}]$ \\
\hline
0.660 & 60.0   & 0.6667 & $\pm 0.017$ \\
0.640 & 62.5   & 0.6400 & $\pm 0.017$ \\
0.620 & 64.0   & 0.6250 & $\pm 0.017$ \\
0.610 & 65.0   & 0.6154 & $\pm 0.017$ \\
0.600 & 67.0   & 0.5970 & $\pm 0.017$ \\
0.580 & 68.0   & 0.5882 & $\pm 0.017$ \\
0.560 & 70.0   & 0.5714 & $\pm 0.017$ \\
0.530 & 75.0   & 0.5333 & $\pm 0.017$ \\
0.510 & 77.5   & 0.5161 & $\pm 0.017$ \\
0.500 & 80.0   & 0.5000 & $\pm 0.017$ \\
\hline
\end{tabular}
\end{table}

\paragraph{Resultados estadísticos calculados}

\[
\overline{I} = 0.581\ \mathrm{A}.
\]

\[
S_{I} = 0.0549\ \mathrm{A}.
\]

\[
\delta_{\overline{I}} = \frac{S_I}{\sqrt{n}} = 0.017\ \mathrm{A}.
\]

Promedio del porcentaje de error relativo:
\[
\overline{\%\mathrm{error}} = 0.839\ \%.
\]

\subsection*{Fase 3 — Resistencia en un material no óhmico}

En esta fase se midió la resistencia efectiva de un bombillo incandescente para diferentes pares de voltaje y corriente. 
Se calculó la resistencia según la Ley de Ohm clásica $R_{\mathrm{Ohm}} = V/I$ y se comparó con la resistencia medida 
$R_{\mathrm{med}}$. La columna ``Diferencia de $R$'' indica la diferencia relativa entre ambas resistencias, expresada en porcentaje. 
La incertidumbre en $R_{\mathrm{med}}$ se obtuvo como el error estándar del promedio de las 10 mediciones.

\[
R_{\mathrm{Ohm}} = \frac{V}{I}, 
\qquad 
\%\mathrm{error} = \frac{|R_{\mathrm{med}} - R_{\mathrm{Ohm}}|}{R_{\mathrm{Ohm}}}\times 100\%.
\]

\begin{table}[h]
\centering
\caption{Fase 3: Mediciones en un material no óhmico. 
(Unidades: $V$ en volts, $I$ en amperes, $R$ en ohmios.)}
\label{tab:fase3}
\begin{tabular}{r r r r r}
\hline
$V\ [\mathrm{V}]$ & $I\ [\mathrm{A}]$ & $R_{\mathrm{med}}\ [\Omega]$ & Error $R$ [\%] & Incertidumbre [\(\Omega\)] \\
\hline
45  & 0.55  & 81  & 1.6   & $\pm 1.5$ \\
50  & 0.58  & 86  & 3.8   & $\pm 1.5$ \\
55  & 0.61  & 90  & 2.4   & $\pm 1.5$ \\
60  & 0.67  & 89  & 2.0   & $\pm 1.5$ \\
65  & 0.66  & 98  & 9.0   & $\pm 1.5$ \\
75  & 0.713 & 105 & 2.0   & $\pm 1.5$ \\
80  & 0.74  & 108 & 2.7   & $\pm 1.5$ \\
85  & 0.76  & 111 & 3.3   & $\pm 1.5$ \\
90  & 0.79  & 113 & 2.0   & $\pm 1.5$ \\
95  & 0.80  & 118 & 5.3   & $\pm 1.5$ \\
\hline
\end{tabular}
\end{table}

\paragraph{Resultados estadísticos}
\[
\overline{R_{\mathrm{med}}} = 99.9\ \Omega,
\qquad
S_{R} = 4.8\ \Omega,
\qquad
\delta_{\overline{R}} = 1.5\ \Omega.
\]

Promedio de la diferencia relativa:
\[
\overline{\%\mathrm{error}} = 3.4\ \%.
\]

\section{Análisis de Resultados}

\section{Conclusiones}

\section{Anexo}

\begin{thebibliography}{9}

\bibitem{Taylor} Taylor, J. R. (1997). \textit{An introduction to error analysis: The study of uncertainties in physical measurements} (2nd ed.). University Science Books.

\bibitem{JCGM} Joint Committee for Guides in Metrology. (2008). \textit{Evaluation of measurement data—Guide to the expression of uncertainty in measurement (JCGM 100:2008)}. International Bureau of Weights and Measures.

\bibitem{Mitutoyo} Mitutoyo Corporation. (2015). \textit{Digital caliper operation manual}. Mitutoyo Corporation.

\end{thebibliography}



\end{document}
